\section{绪论}

\begin{definition}[绝对误差]
设 $x$ 为准确值, $x^{*}$ 为 $x$ 的一个近似值, 称 $e^{*}=x^{*}-x$ 为近似值的绝对误差
\end{definition}

\begin{definition}[相对误差]
我们把近似值的误差 $e^{*}$ 与准确值 $x$ 的比值
$$
\frac{e^{*}}{x}=\frac{x^{*}-x}{x}
$$
称为近似值 $x^{*}$ 的相对误差, 记作 $e_{r}^{*}$.
\end{definition}

\begin{definition}[相对误差限]
  $\varepsilon_{r}^{*}=\frac{\varepsilon^{*}}{\left|x^{*}\right|}$
\end{definition}

\begin{definition}[有效数字]
若近似值 $x^{*}$ 的误差限是某一位的半个单位, 该位到 $x^{*}$ 的第一位非零数字共 有 $n$ 位, 就说 $x^{*}$ 有 $n$ 位有效数字. 它可表示为
\begin{equation}
x^{*}=\pm 10^{m} \times\left(a_{1}+a_{2} \times 10^{-1}+\cdots+a_{n} \times 10^{-(n-1)}\right),
\end{equation}
其中 $a_{i}(i=1,2, \cdots, n)$ 是0 到9 中的一个数字, $a_{1} \neq 0, m$ 为整数, 且
\begin{equation}
\left|x-x ^{*}\right| \leq \frac{1}{2} \times 10^{m-n+1} .
\end{equation}
\end{definition}

\begin{theorem}
设近似数 $x^{*}$ 表示为
\begin{equation}
x^{*}=\pm 10^{m} \times\left(a_{1}+a_{2} \times 10^{-1}+\cdots+a_{l} \times 10^{-(l-1)}\right),
  \label{eq:1}
\end{equation}
其中 $a_{i}(i=1,2, \cdots, l)$ 是 0 到 9 中的一个数字, $a_{1} \neq 0, m$ 为整数. 若 $x^{*}$ 具有 $n$ 位有效数字, 则其相对误差限
$$
\varepsilon_{r}^{*} \leq \frac{1}{2 a_{1}} \times 10^{-(n-1)}
$$
反之, 若 $x^{*}$ 的相对误差限 $\varepsilon_{r}^{*} \leq \frac{1}{2\left(a_{1}+1\right)} \times 10^{-(n-1)}$, 则 $x^{*}$ 至少具有 $n$ 位有效数字.
\end{theorem}

\begin{proof}
由(\ref{eq:1})式可得
$$
a_{1} \times 10^{m} \leq \left|x^{*}\right|<\left(a_{1}+1\right) \times 10^{m}
$$
当 $x^{*}$ 具有 $n$ 位有效数字时
$$
\varepsilon_{r}^{*}=\frac{\left|x-x^{*}\right|}{\left|x^{*}\right|} \leq \frac{0.5 \times 10^{m-n+1}}{a_{1} \times 10^{m}}=\frac{1}{2 a_{1}} \times 10^{-n+1}
$$
反之,由
$$
\begin{aligned}
\left|x-x^{*}\right| &=\left|x^{*}\right| \varepsilon_{r}^{*}<\left(a_{1}+1\right) \times 10^{m} \times \frac{1}{2\left(a_{1}+1\right)} \times 10^{-n+1} \\
&=0.5 \times 10^{m-n+1},
\end{aligned}
$$
故 $x^{*}$ 至少具有 $n$ 位有效数字. 证毕. 
\end{proof}

\begin{corollary}
已知 $x^{*}$ 的相对误差限可写为 $\varepsilon_{r}^{*}=\frac{1}{2\left(a_{1}+1\right)} \times 10^{-n+1}$
,则 $\left|x-x^{*}\right| \leq \varepsilon_{r}^{*} \cdot\left|x^{*}\right|=\frac{10^{-n+1}}{2\left(a_{1}+1\right)} \times 0 . a_{1} a_{2} \cdots \times 10^{m}$
$$
<\frac{10^{-n+1}}{2\left(a_{1}+1\right)} \cdot\left(a_{1}+1\right) \times 10^{m-1}=0.5 \times 10^{m-n}
$$
可见 $x^{*}$ 至少有 $n$ 位有效数字。 
\end{corollary}
